\documentclass{sice-si}

% タイトルと著者名
\title{視覚と行動のend-to-end学習により経路追従行動を\\
オンラインで模倣する手法の提案\\
トポロジカルマップとシナリオに基づく経路選択機能の追加と検討\\} % 和文タイトル
\name{○春山 健太(千葉工大),藤原 柾(千葉工大)} % 著者名
\etitle{Instruction for SICE SI Annual Conference Manuscript} % 英文タイトル
\ename{○Kenta HARUYAMA (CIT), and Masaki Fujiwara (CIT)}	%著者名(英)

\begin{document}
% アブストラクト
\abst{
    This manuscript describes a method for preparing a manuscript for the annual conference of the SICE SI division.
}

% タイトルの出力
\maketitle

% 本文
\section{緒言}

本研究グループでは,end-to-end 学習により,カメラ画像を
入力として,経路を追従する行動をオンラインで模倣する手
法を提案してきた[岡田][岡田][清岡][高橋][今井],
% \cite{okada2020}
[春山][藤原]では,これに経路を選択する機能を追加と,成功率の不均衡データの緩和,
積極的な蛇行により学習時間を短縮する手法を行っている.
この手法(以後,本手法とする)をシミュレータや実ロボットを用いた
実験により有効性を検証した,本手法では,〜に示すような地図に基づく経路追従行動を模
倣して,〜のようなカメラ画像を入力とする経路追従行動
を生成する.さらに,分岐路などで目標とする進行方向の情報(以後,目標方向
と呼ぶ)に応じて,経路を選択して走行する.\par
本手法により,
地図に基づく経路追従とカメラ画像を入力とする経路追従の
2 つのナビゲーション手段が得られる.この 2 つの手段を状況
に応じて高い信頼性が見込まれる方を選択することで,
経路追従を継続できる可能性が高まる.
[春山]や[藤原]では,カメラ画像から目標方向を生成する方法をいない.
そのため,学習後のカメラ画像を入力とする経路追従において,
目標方向の生成を,学習時と同様に地図に基づいた制御器から行っていた.
この地図に基づいた制御器への依存を解決のために[春山藤原]で述べたように
シナリオに基づいたナビゲーション[島田〜原]との統合を行う.
これにより,目標方向の生成,経路追従をカメラのみで行い,設定された経路に従って
自律移動することが期待される.
\par
本稿では,主にカメラ画像を入力とする経路追従に対して,
カメラ画像から目標方向の生成し,経路の指示を行う
ナビゲーションの追加について議論する.
また,実ロボットを用いた実験を通して,構築したシステムの有効性を確認する.

\section{提案手法}
本手法では,カメラ画像と「条件」と「行動」の単語による経路の表現に基づいて,
目的地まで自律移動する.
ここではまず初めに全体の流れについて述べた後,
それぞれの詳細を述べる.
システムの概要を〜に示す.
システムは,センサ入力はRGBカメラとし,
主に
1)カメラ画像を入力とする通路分類器
2)シナリオに基づいたナビゲーション
3)経路選択機能を持つ学習器
の3つのモジュールで構成されている.

2)に対してシナリオを入力し,経路を設定を行う.
カメラから得た画像データを1)の通路分類器に入力する.
2)は通路分類器が出力する通路の特徴を基に,目標方向を生成する.
3)は2)から生成された目標方向とカメラから得た画像データを基に,
ヨー方向の角速度を出力する.
ロボットは3)から出力された角速度を基に経路を追従する.
\subsection{用紙サイズ,書式など}
\subsubsection{原稿の体裁}
用紙サイズはA4版(縦297mm$\times$横210mm)とし,余白部分は左右15mm,上20mm,下27mmを確保してください.(プログラム委員会でヘッダ・フッダ部分に情報を追加する予定ですので,ご注意ください.)
よって,原稿作成領域は250mm$\times$180mmの枠内となります.

\subsection{経路選択機能を持つ学習器}
学習器のシステムを〜に示す.学習段階では,2D-LiDARやオドメトリ,占有格子地図に基づく,
地図ベースの制御器によって,設定した経路を走行する.
その際,入力をカメラ画像,目標方向,
地図ベースの制御器が出力するヨー方向の角速度を出力として,データセットに加える.
さらに,設定したバッチサイズ分の教師データをデータセットから抽出し,
オンラインで模倣学習を行う.この1連の流れを1ステップとする.
データセットを収集するには,[藤原]で提案された,データセットに加える不均衡の改善,
学習時における積極的な蛇行といった最も成功率の高い手法を用いる.\par
学習後,訓練した学習器に対してカメラ画像と目標方向を入力し,ロボットを制御する.
これまで,[春山][藤原]では学習後も,目標方向を地図ベースの制御器から出力していたが,
本項では,次の小節で述べるシナリオに基づくナビゲーションから出力を行う.
なお学習器に用いるネットワークは[春山][藤原]と同様の構成を用いる.

\subsection{シナリオを用いたナビゲーション}
目的地までの経路の設定および,目標方向の生成を行う通路の特徴に基づいた
ナビゲーションについて述べる.
この手法は[島田ら]で提案された人の道案内情報を収集・分析し,
そのデータを基にトポロジカルマップの形式とシナリオ(経路の表現)を用いて
ロボットをナビゲーションする手法をベースに構成する.
% ~にトポロジカルマップとシナリオの例を示す.

目標方向の生成は[島田]で開発された
1)シナリオからロボットの制御用の手順を生成する機能を拡張して行う.
「3つ目の三叉路まで直進.右折.突き当たりまで直進.停止」
というシナリオの例を基に1)の機能と目標方向の生成について述べる.
シナリオは句点ごとに分解後,「条件」と「行動」を示す言葉を抽出し,
以下の項目に分けて登録する.\\
1)通路の特徴\\
2)順番\\
3)方向\\
4)行動\\
シナリオの例は
3つ目の三叉路まで直進
右折
突き当たりまで直進
停止
と句点ごとに区切られ.
1つ目の区切りでは
1)通路の特徴:三叉路 
2)順番:3 
4)行動:直進 
2つ目の区切りでは
4)行動:右折
が登録される
この一連の作業を最後の停止まで行う.
が登録される.
ここで登録した「行動」を〜に示すワンホットベクトルで表現し,
目標方向として,〜で述べた学習器へ与える.

\begin{table}[]
    \centering
    \caption{target}
    \begin{tabular}{cc}
    \hline
    目標方向 & ベクトル        \\
    \hline
    直進   & {[}1,0,0{]} \\
    左折   & {[}0,1,0{]} \\
    右折   & {[}0,0,1{]} \\
    停止   & {[}0,0,0{]}\\
    \hline
    \end{tabular}
    \end{table}

\par
島田らは1)の昨日の他に2)通路の特徴を検出する機能
3)経路に沿って通路を走行する3つの機能を開発している.
しかし,[島田][原]の手法では2)と3)にLiDARや全天球カメラのセンサ入力を
必要としている.
そのため本項では.
2)の通路の特徴検出は,次の小節で述べるカメラ画像による手法,
3)に関しては〜で述べた学習器へ変更している.
\subsection{カメラ画像を用いた通路分類}
カメラ画像と機械学習を用いて,通路分類を行う通路分類器について述べる.
通路分類器の概要を〜に示す.通路分類器は
シーケンスの画像データを入力とし,それに対する通路の特徴の分類を出力する.
通路の特徴の分類は[島田]に倣い,〜に示した8つとする.

\par
具体的な通路分類器のネットワーク構造を図〜に示す.
構造に関してD.バットらが提案するCNNとLSTMを組み合わせた
LRCNをアーキテクチャの参考としている.
CNNには実行速度の観点からMobileNetV3-Largeを用いる.
フレーム数は16 入力画像サイズを64x48 出力8としている.

学習するデータセット内で,クラス間のデータ数が大きく異なる不均衡データは,
分類に大きな影響を与えるとされている.[分類]
そのため,本稿では学習する際に,データセット内のクラス間のデータ数によって
重み付けを行うコストアプローチを導入している.[コストアプローチ]

\section{実験}
経路を選択する機能を持つ学習器に対して,シナリオを用いたナビゲーションから
生成した目標方向をあたえ,カメラ画像に基づいて指示された経路に従い,
目的地へ到達可能であるか検証する実験を行う.

\subsection{実験装置}
実験には〜でしめす本学で開発しているorne gammaをベースに,
カメラを3つ搭載したロボットを使用する.

\subsection{実験方法}
実験環境として〜で示した千葉工業大学津田沼キャンパス2号館3階の廊下を用いる.

まず初めに通路分類器の訓練を行う.
〜に示した経路を地図ベースの制御器の出力で用いて,2周し,データセットを収集する.
1周目のデータを訓練データとし,2周目のデータをテストデータとする.
訓練はバッチサイズを32として30epoch行った.
次に経路選択機能を持つ学習器の訓練を行う.
通路分類器の訓練と同様の経路を,オンラインで模倣学習しながら1周走行する.
その際のステップ数は〜である.
実験では島田らが用いた50例のシナリオの中から,
図中に示したエリアを対象とした7例を抽出して用いる.
その際,~のように1.地図ベースの制御器で通行が困難な場所が含まれるもの.
〜のように2.その場で「右を向く」といった学習器の出力を用いた走行では達成が困難なもの
を除外している.
抽出した7例のうちの1例のシナリオ,
また訓練した通路分類器と経路選択機能を持つ学習器をシステムにセットする.
その後,シナリオのスタート地点へシナリオに基づいた向きでロボットを配置し,
実験を開始する.
なお,経路から外れるといった要因で走行が困難になった場合でも即時失敗とせず,
失敗箇所を記録しながら,人間が介入し,実験を継続する.
〜に実験で用いたシナリオ例を示す.
今回は50例のシナリオのうち,2号館3階の一部のエリアの対象とした例に限定しているが,
今後はフロア全体を対象して拡張する予定である.

\subsection{実験結果}
実験結果を〜に示す.
表はそれぞれ実験に用いたシナリオの番号,学習器の出力が要因による介入の回数,
通路分類の間違いを原因とする介入の回数である.
結果として,抽出した7例のうち,7例すべての例で人間の介入なしで,
ロボットが指定された経路を追従して目的地へ到達した.
以上の結果から,経路選択機能を持つ学習器に対して,
追加したシナリオに基づくナビゲーションとカメラ画像による通路分類を追加したシステムが,
適切に動作することを確認した.
\begin{table}[]
    \centering
    \caption{実験結果}
    \begin{tabular}{ccclll}
    \hline
    シナリオNo. & 介入回数(学習器) & 介入回数(通路分類) \\
    \hline
    1       & 0         & 0             \\
    5       & 0         & 0             \\
    20      & 0         & 0             \\
    21      & 0         & 0             \\
    22      & 0         & 0             \\
    24      & 0         & 0             \\
    50      & 0         & 0             \\
    \hline
    \end{tabular}
    \end{table}

\section{結言}
本稿では,経路選択機能を持つ学習器を用いた走行に対して,
カメラ画像からの目標方向の生成を目的として,シナリオを用いたナビゲーションと通路分類器の
追加を行った.
実験からシナリオとカメラ画像を用いて目標方向の生成を行い,
学習器がカメラ画像と生成された目標方向を用いて,
指定された経路に沿って目的地へ到達可能であることを確認した.
% この7例では,模倣学習側,および通路検出の双方で人間の介入なしで目的地へ到達.

% \subsubsection{基本書式}
% 原稿の記載内容は,下記の順序とします.
% \begin{enumerate}[label=\arabic*), labelsep=1em]
%     \item 和文題目(英文原稿の場合には不要,16ptゴシックフォント推奨,センタリング)
%     \item 和文著者名・所属(英文原稿の場合には不要,12pt明朝フォント推奨,センタリング,登壇者に○を付加)
%     \item 英文題目(16pt TeX Gyre Termes Bold推奨,センタリング)
%     \item 英文著者名・所属(12pt TeX Gyre Termes推奨,センタリング,登壇者に○を付加)
%     \item 英文アブストラクト(9pt TeX Gyre Termes推奨,3 〜 5行程度,文章両側を10mm程度インデント)
%     \item 本文(本文文章は10pt明朝フォント推奨,小見出しは12 〜 10pt程度のゴシックフォント推奨)
%     \item 参考文献(10pt明朝フォント推奨)
% \end{enumerate}
% \subsubsection{図と表について}
% 予稿はPDFファイルとなりますので,図や表はカラーで作成していただいても構いません.
% ただしファイルサイズの制限にご注意ください.
% 図のキャプションは図の下にFig.1,Fig.2という具合に,表のキャプションは表の上にTable 1,Table 2という具合にお付けください.(英語表記,フォントは10pt TeX Gyre Termes推奨)

% \section{結言}
% 本稿はあくまでも予稿原稿を作成するためのガイドラインを示したものです.改行幅やフォントの設定などについては,原稿の内容や量に合わせて適宜判断していただき,原稿を作成してください.
% また,本稿はSICE-SIの予稿原稿の書き方\cite{SI}\cite{SIbook}\cite{WebPage}を参考に,\TeX 用書式を用意したものです.適宜sice-si.clsを変更して使用してください.


% %参考文献
% % \begin{thebibliography}{99}
% %     \bibitem{SI}
% %     計測太郎,制御花子:
% %     ``SICE SI予稿原稿の書き方(サンプル)'',
% %     {\it 計測自動制御学会SI部門講演会SICE-SI予稿集},
% %     pp.0000--0000 (20??)
% % \end{thebibliography}

\printbibliography[title=参考文献]

\end{document}
